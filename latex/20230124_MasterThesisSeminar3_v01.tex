\documentclass[9pt]{beamer}
\usepackage[utf8]{inputenc}
\usepackage{floatrow}

\usepackage{graphicx}
\graphicspath{{./graphs/}}	% graphics

\title{Master thesis presentation 3}
\author{Grahl, Lukas}
\institute{Paris 1: Panthéon Sorbonne}
\date{\today}

\bibliographystyle{plain}
\DeclareFloatFont{tiny}{\tiny}
\floatsetup[table]{font=tiny}

\begin{document}
	
	\maketitle
	
	\begin{frame}{Data Overview}
		
	To this date I have obtained the following data.
		
		\begin{center}
		\tiny{
		\begin{tabular}{l|llll}
			Data & Purpose & Source & Status & Frequency \\
			\hline
			Inflation $\pi$& realised inflation & Eurostat & obtained & M \\
			Infl. expectation $\pi^e$ & consumer inflation expectation & Bundesbank & obtained & M \\
			ECB speeches & Supplement news data & ECB & obtained & 6-W \\
			News-paper articles & Analyse inflation reporting & Bundesbank & waiting & D \\
			
		\end{tabular}
		}	
		\end{center}	
	
	While I am waiting on the news paper data I am currently preparing my analysis code using ECB press releases.
	
	Moreover, in the upcoming weeks I will be working on the opinion formation model as presented last time. 
	
	\end{frame}

	\begin{frame}{Identifying surprise inflation}
	Pre-pandemic inflation behaviour can be describe by an intercept $\alpha$, deterministic trend $\beta_1t$ and seasonality $s_{t|s}$.
	\[
		CPI_t - CPI_{t-1} = \pi_t = \alpha + \beta_1 t + \beta_2 s_{t|s} + \epsilon_t
	\]
	The post-pandemic up-tick in inflation can be identified as a divergence from this model. This strategy only holds for a limited period.
		\begin{center}
			\includegraphics[scale=.4]{presentation3_graph1.png}
		\end{center}
	\end{frame}
	
	\begin{frame}{Testing surprise inflation for white noise}
	The above model only is a good model if its residual $\epsilon_t$ is white noise. The test for normality suggests it is.
	The post-pandemic deviation from the model is not white noise. It may thus be interpreted as surprise inflation from the consumer's point of view.
		\begin{center}
			\includegraphics[scale=.4]{presentation3_graph2.png}
		\end{center}
	\end{frame}

	\begin{frame}{Consumer inflation expectation}
		
	Comparing inflation expectation and surprise as well as realised inflation a pattern appears. Inflation expectation lag behind realised inflation. 
	Spikes in surprise inflation do not seem to affect expectation by much.
		\begin{center}
			\includegraphics[scale=.2]{presentation3_graph3.png}
		\end{center}
	The odd shape of the violin plot stems from frequent indication of discrete round inflation values. A nice example of the round-number bias.
	
	\end{frame}

	
	\begin{frame}{ECB reaction to inflation}
	
	Unsurprisingly ECB speeches have been for a long-time concerned with inflation. This is potentially because of previously too low inflation as well as its recent surge.
	
	Analysis of news-paper articles will likely present another picture. I suspect inflation mentions to only be recently on the rise. 
		\begin{center}
			\includegraphics[scale=.4]{presentation3_graph4.png}
		\end{center}
	
	As validation of the Term-Frequency-Inverse-Document-Frequency (TF-IDF) I have provided other topics, likely related to pas and current crisis.
	\end{frame}


\end{document}