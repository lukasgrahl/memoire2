\documentclass[9pt]{beamer}
\usepackage[utf8]{inputenc}
\usepackage{floatrow}

\usepackage{graphicx}
\graphicspath{{./graphs/}}	% graphics

\title{Master thesis presentation 3}
\author{Grahl, Lukas}
\institute{Paris 1: Panthéon Sorbonne}
\date{\today}

\bibliographystyle{plain}
\DeclareFloatFont{tiny}{\tiny}
\floatsetup[table]{font=tiny}

\usepackage{fancyvrb}
\RecustomVerbatimCommand{\VerbatimInput}{VerbatimInput}%
{fontsize=\footnotesize,
	%
	frame=lines,  % top and bottom rule only
	framesep=2em, % separation between frame and text
%	rulecolor=\color{Gray},
	%
	%	label=\fbox{\color{Black}data.txt},
	labelposition=topline,
	%
	commandchars=\|\(\), % escape character and argument delimiters for
	% commands within the verbatim
	commentchar=*        % comment character
}


\begin{document}
	
	\maketitle
	
	\begin{frame}{Time line and completion overview}
		
		To this date I have obtained the following data.
		\begin{center}
			\tiny{
				\begin{tabular}{l|llll}
				Data & Purpose & Source & Status & Frequency \\
				\hline
				Inflation $\pi$& realised inflation & Eurostat & obtained & M \\
				Infl. expectation $\pi^e$ & consumer inflation expectation & Bundesbank & obtained & M \\
				ECB speeches & Supplement news data & ECB & obtained & 6-W \\
				Break-even inflation & market measure of inflation expectation & Reuters & obtained & D \\
				News-paper articles & Analyse inflation reporting & Bundesbank & waiting & D \\							
				\end{tabular}
			}	
		\end{center}	
		Presenting today: 
		\begin{itemize}
				\item Preprocessing of each time series (trend, seasonality, etc.)
				\item Analysis of consumer inflation expectation residual
		\end{itemize}
	
		Next steps
		\begin{itemize}
			\item Prepare news-paper data analysis, extracting inflation narrative occurrences
			\item Run VAR of different narratives on inflation expectation
			\item Derive the probabilistic opinion formation model as presented last time
			\item Bayesian estimation of opinion formation model
		\end{itemize}
		
	\end{frame}

	\begin{frame}{Preprocessing: Descriptive Statistics}
		\VerbatimInput{{./graphs/desc.txt}}
	\end{frame}

	\begin{frame}{Preprocessing: Covariance matrix}
		\VerbatimInput{{./graphs/cov.txt}}
	\end{frame}

	\begin{frame}{Preprocessing: Identifying surprise inflation}
	
		Pre-pandemic inflation follows a stable path, which can bed describe by an intercept $\alpha$, deterministic trend $\beta_1t$ and seasonality $s_{t|s}$.
		\[
		CPI_t - CPI_{t-1} = \pi_t = \alpha + \beta_1 t + \beta_2 	s_{t|s} + \epsilon_t
		\]
		The post-pandemic up-tick in inflation can be identified as a divergence from this model. Such strategy, evidently, only holds for a limited period.
		
		\begin{center}
			\includegraphics[scale=.4]{presentation3_graph1.png}
		\end{center}
		
	\end{frame}


	
	\begin{frame}{Testing surprise inflation for white noise}

		The above model only is a good model if its residual $\epsilon_t$ is white noise. The test for normality suggests it is.
		The post-pandemic deviation from the model is not white noise. It may thus be interpreted as surprise inflation from the consumer's point of view.
		
		\begin{center}
			\includegraphics[scale=.4]{presentation3_graph2.png}
		
		\end{center}
	
	\end{frame}
	
	\begin{frame}{Preprocessing: ECB reaction to inflation}
		
		Unsurprisingly ECB speeches have been for a long-time concerned with inflation. This is potentially because of previously too low inflation as well as its recent surge.
		
		Analysis of news-paper articles will likely present another picture. I suspect inflation mentions to only be recently on the rise.
		
		\begin{center}
			\includegraphics[scale=.4]{presentation3_graph4.png}
		\end{center}
		
		As validation of the Term-Frequency-Inverse-Document-Frequency (TF-IDF) I have provided other topics, likely related to pas and current crisis.
		
	\end{frame}

	
	\begin{frame}{Preprocessing: Break Even Inflation (BEI)}
		
		Break-even inflation for Germany is the difference between the yield to maturity (YTM) of a 10y government bond and its inflation adjusted counterpart.
		
		I obtained the YTM through Newton's method solver. The results match up with the BEI for France, obtained from the Banque de France.
		
		\begin{center}
			\includegraphics[scale=.275]{presentation3_graph5.png}
		\end{center}
		
		Break-even inflation, though different in levels appear to follow a similar pattern to surprise inflation.
		
	\end{frame}

	\begin{frame}{Preprocessing: Consumer inflation expectation}
		
		The household panel inflation expectation survey inquires about inflation expectation of thousands of households every month. Most households are surveyed up to 20 times in two month windows. 
		I am therefore estimating average inflation expectation from the point expectation by household (red dots) using spline Bayesian estimation. \\
				
		In comparing inflation expectation with surprise and realised inflation a pattern emerges. Inflation expectation underestimates realised inflation. 
		
		\begin{center}
			\includegraphics[scale=.3]{presentation3_graph6.png}
		\end{center}
	
	\end{frame}

	\begin{frame}{Analysis: Consumer inflation expectation}
		
	The above provides an estimate for overall inflation expectation. It does not provide an account for the individual errors of agents.
	An investigation leveraging the panel dimension reveals that less precise agents tend to over rather than underestimate inflation.
	
		\[
			\pi^s_t = \alpha_i + \beta_i \pi^e_{i,t} + \epsilon_{i,t} \quad \hat{\pi}_{i,t}^e = \pi^s_t - \alpha_i + \beta_i \pi^e_{i,t}
		\]
		
		\[
			mse_i = \sum_{i=0}^{I} \left[\pi^s_t - \hat{\pi^e_{i,t}} \right]^2
			\quad
			me_i =  \sum_{i=0}^{I} \left[\pi^s_t - \hat{\pi^e_{i,t}} \right]
		\]
		
		\begin{center}
			\includegraphics[scale=.3]{presentation3_pi_resid.png}
		\end{center}
	\end{frame}

	

	

	


\end{document}